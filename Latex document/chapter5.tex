%----------------------------------------------------------------------------
\chapter{Experimental results}\label{chap:Tests}
%----------------------------------------------------------------------------
In the following chapter the results of the system tests will be discussed. 
Performance and precision criteria 
%----------------------------------------------------------------------------
\section{Evaluation results}
%----------------------------------------------------------------------------
\subsection{Test videos}

\begin{table}[]
		%\centering
	\begin{adjustbox}{center}
\begin{tabular}{|l|l|l|l|l|}
	\hline
	\multicolumn{1}{|c|}{\textbf{Video code}} & \multicolumn{1}{c|}{\textbf{Characteristics}}                                                     & \multicolumn{1}{c|}{\textbf{Lanes set}} & \multicolumn{1}{c|}{\textbf{\begin{tabular}[c]{@{}c@{}}Number of\\ GT\\  points\end{tabular}}} & \multicolumn{1}{c|}{\textbf{Special features}}                                                \\ \hline \hline
	\textbf{BorA1}                            & night, pedestrians                                                                                & no                                      & 14                                                                                             & minSize = 20                                                                                  \\ \hline
	\textbf{BorA2}                            & \begin{tabular}[c]{@{}l@{}}daytime, pedestrians, \\ stopping tram\end{tabular}                    & no                                      & 13                                                                                             & minSize = 10                                                                                  \\ \hline
	\textbf{ErA1}                             & night, low traffic                                                                                & yes                                     & 8                                                                                              & --                                                                                            \\ \hline
	\textbf{ErA2}                             & daytime, low traffic                                                                              & yes                                     & 59                                                                                             & --                                                                                            \\ \hline
	\textbf{ErC1}                             & \begin{tabular}[c]{@{}l@{}}daytime, stopping tram, \\ pedestrians\end{tabular}                    & yes                                     & 45                                                                                             & minSize = 40                                                                                  \\ \hline
	\textbf{ErE5}                             & \begin{tabular}[c]{@{}l@{}}recorded at the \\ IBikeBudapest cyclist \\ demonstration\end{tabular} & no                                      & 202                                                                                            & minSize = 10                                                                                  \\ \hline
	\textbf{ErzsC\_N}                         & \begin{tabular}[c]{@{}l@{}}night, low traffic, \\ some pedestrians\end{tabular}                   & yes                                     & 18                                                                                             & --                                                                                            \\ \hline
	\textbf{ErzsC\_S}                         & \begin{tabular}[c]{@{}l@{}}daytime, many\\ pedestrians, occlusion\end{tabular}                    & no                                      & 18                                                                                             & \begin{tabular}[c]{@{}l@{}}MOG2 overreaction \\ detection: higher \\ sensitivity\end{tabular} \\ \hline
	\textbf{ErzsD}                            & daytime, low traffic                                                                              & yes                                     & 32                                                                                             & --                                                                                            \\ \hline
	\textbf{FovamC\_D}                        & sundown, low traffic                                                                              & yes                                     & 79                                                                                             & --                                                                                            \\ \hline
	\textbf{FovamC\_N}                        & \begin{tabular}[c]{@{}l@{}}night, low traffic, \\ stopping vehicles\end{tabular}                  & no                                      & 38                                                                                             & --                                                                                            \\ \hline
	\textbf{FovC1}                            & sundown, low traffic                                                                              & yes                                     & 95                                                                                             & --                                                                                            \\ \hline
	\textbf{GoldA1}                           & night, medium traffic                                                                             & yes                                     & 13                                                                                             & --                                                                                            \\ \hline
	\textbf{GoldA2}                           & \begin{tabular}[c]{@{}l@{}}daytime, heavy traffic, \\ stopping vehicles\end{tabular}              & yes                                     & 68                                                                                             & calibration corrected                                                                         \\ \hline
	\textbf{GoldB1}                           & \begin{tabular}[c]{@{}l@{}}daytime, heavy traffic, \\ stopping vehicles\end{tabular}              & no                                      & 46                                                                                             & calibration corrected                                                                         \\ \hline
	\textbf{GoldB2}                           & daytime, low traffic                                                                              & yes                                     & 10                                                                                             & --                                                                                            \\ \hline
	\textbf{SasA1}                            & daytime, medium traffic                                                                           & yes                                     & 194                                                                                            & --                                                                                            \\ \hline
	\textbf{SasadA1}                          & daytime, heavy traffic                                                                            & yes                                     & 195                                                                                            & \begin{tabular}[c]{@{}l@{}}MOG2 overreaction \\ detection: higher \\ sensitivity\end{tabular} \\ \hline
\end{tabular}
\end{adjustbox}
\caption{My caption}
\label{my-label}
\end{table}

\subsection{Results}
\subsubsection{Initial setting}
- minSize: 250
- lanes: defined for most videos

\subsubsection{The minimal size of blobs redefined}
- minSize: 200
- lanes: defined for most videos

\subsubsection{No lanes defined}
- minSize: 200
- lanes: no lanes

\subsubsection{Split size added}
- minSize: 100
- splitMinSize : 200

\subsection{Common errors and solutions}
\subsubsection{False positives}
\subsubsection{False negatives}
\subsubsection{Misclassification errors}

%----------------------------------------------------------------------------
\section{Calibration correction}\label{chap:cal_corr}
%----------------------------------------------------------------------------
Due to perspective distortion, vehicles, that in effect have the same sizes differ in length on the Raw Frames.
Since vehicle lengths are the basis of classification, perspective distortion is eliminated using a quadrangle, called the calibration rectangle, that is defined by the user.

The calibration rectangle is hand-drawn on a frame in the Project Configurator GUI (section \ref{chap:calibration}.) as part of the calibration process.
The position and the sides' lengths are estimated with the help of any occurring road lanes, or simply based on naked eye examination.
On this wise, due to the inaccuracy of human perception, the location and size of the calibration rectangle are significantly uncertain.
Empirical analysis shows that errors are actually present in many hand-calibrated videos.

To avoid manual calibration, some traffic surveillance systems automatize this process.
However in heterogeneous environment, it is a highly complex problem, and is not implemented in the current version of the Traffic Sensor system.

To correct the existent settings, the extent of the miscalibration is evaluated examining the length of the vehicles close to and far from the camera.
This way the calibration rectangle is redefined for later use during the evaluation phase of the processing.

\subsection{The miscalibration problem}
\subsubsection{Perspective compensation method}
The calibration rectangle is constructed by defining its four edges from the upper left corner, to the lower left one in order.
The edges must be coplanar in the plane of the street, and correspond to a square's vertices in the 3D world.
The square, due to the perspective transformation, shows as a general rectangle on the image plane.
On most frames, lane markings are used to find and define the square, since they are parallel lines in the road.

The four vertices are then projected back into a square, along with the entire pixel grid of the image using a transformation map to speed up the process.
Examples of the Frames before and after the perspective compensation transformation are seen in figure \ref{fig:perscomp}.
For defining the transform and constructing the map, universal OpenCV methods were used\cite{PersTrans, WrapPers}.

\begin{figure}[!t]
	\centering
		\begin{subfigure}[b]{0.4\textwidth}
			\includegraphics[width=\textwidth]{raw_frame_ibike.png}
			\caption{A Raw Frame from the ErE5 video sequence.}
		\end{subfigure}
		\quad
		\begin{subfigure}[b]{0.4\textwidth}
			\includegraphics[width=\textwidth]{original_frame_ibike.png}
			\caption{An Original Frame from the ErE5 video sequence.}
		\end{subfigure}
		\hfill
		\begin{subfigure}[b]{0.4\textwidth}
			\includegraphics[width=\textwidth]{raw_frame_fovam.png}
			\caption{A Raw Frame from the FovamC\_D video sequence.}
		\end{subfigure}
		\quad
		\begin{subfigure}[b]{0.4\textwidth}
			\includegraphics[width=\textwidth]{original_frame_fovam.png}
			\caption{An Original Frame from the FovamC\_D video sequence.}
		\end{subfigure}
		
		\caption{Frames before and after the perspective compensation. The calibration rectangle's edges are marked with their sequential numbers.\label{fig:perscomp}}
\end{figure}

In the result images -- the Original Frame-strip -- the vehicle lengths are independent of the vehicle's position in the image plane. 
The calibration rectangle's sides are also used to define a scaling factor between the 3D world and the image plane.
Using the scaling factor all lengths are converted to a metric scale, thus correct classification is possible.

\subsubsection{Miscalibrated streams}
A miscalibration is indicated with gradual decrease of vehicle lengths as moving further from the camera.
It is caused by the edges of the rectangle that do not fit a perfect square in the 3D world.
- the vehicles on the compensated frames are not the same length \ref{fig:calibrations}
- in most cases their lengths still decrease as (távolodva) from the camera
- easily (felismerheto) (jelenseg): the int the farest?? road lane cares are classified as bicycles

\begin{figure}[!t]
	\centering
	\begin{subfigure}[b]{0.4\textwidth}
		\includegraphics[width=\textwidth]{calibrationCorrection_OK_GoldB1_Sun.png}
		\caption{GoldB1, correctly calibrated.}
	\end{subfigure}
	\quad
	\begin{subfigure}[b]{0.4\textwidth}
		\includegraphics[width=\textwidth]{calibrationCorrection_GoldA2_Not_OK.png}
		\caption{FovamC\_D, miscalibrated.}
	\end{subfigure}

	\caption{Average vehicle lengths along the Tripwire calculated from a correctly calibrated and a miscalibrated Frame-strip. If the video stream is correctly calibrated, average sizes do not change along the Tripwire. The sizes show a slight decrease, if the video is miscalibrated.(The light blue line indicates the estimated slope of the length values.)\label{fig:calibrations}}
\end{figure}

\subsection{Solutions}
\begin{enumerate}
	\item  \textbf{Estimate decrease}
	\item  \textbf{Calculate proper size ratios}
	\item  \textbf{Redraw calibration rectangle}
\end{enumerate}
- quantify the decrease of the lengths
- based on the correct lengths, redraw the calibration rect

\subsection{Calibration correction results}
\ref{fig:calibrations2}

\addtolength{\topmargin}{-.6in}
\begin{figure}[p]
	\thispagestyle{empty}
	\centering
	\thisfloatpagestyle{empty}
	\begin{subfigure}[t]{0.36\textwidth}
		\includegraphics[width=\textwidth]{calibrationCorrection_GoldA2_Not_OK.png}
		\caption{GoldA2, miscalibrated.}
	\end{subfigure}
	\quad
	\begin{subfigure}[t]{0.36\textwidth}
		\includegraphics[width=\textwidth]{calibrationCorrection_OK_GoldA2.png}
		\caption{GoldA2, corrected.}
	\end{subfigure}
	\hfill
	\begin{subfigure}[t]{0.36\textwidth}
		\includegraphics[width=\textwidth]{calibrationCorrection_Fovam_not_OK.png}
		\caption{FovamC\_D, miscalibrated.}
	\end{subfigure}
	\quad
	\begin{subfigure}[t]{0.36\textwidth}
		\includegraphics[width=\textwidth]{calibrationCorrection_Fovam_OK.png}
		\caption{FovamC\_D, corrected.}
	\end{subfigure}
\hfill
\begin{subfigure}[t]{0.36\textwidth}
	\includegraphics[width=\textwidth]{calibrationCorrection_OK_SasA1_Med.png}
	\caption{SasA1, miscalibrated.}
\end{subfigure}
\quad
\begin{subfigure}[t]{0.36\textwidth}
	\includegraphics[width=\textwidth]{calibrationCorrection_not_OK_SasA1_Med.png}
	\caption{SasA1, corrected.}
\end{subfigure}
\hfill
\begin{subfigure}[t]{0.36\textwidth}
	\includegraphics[width=\textwidth]{calibrationCorrection_not_OK_GoldB1_Sun.png}
	\caption{GoldB1, miscalibrated.}
\end{subfigure}
\quad
\begin{subfigure}[t]{0.36\textwidth}
	\includegraphics[width=\textwidth]{calibrationCorrection_OK_GoldB1_Sun.png}
	\caption{GoldB1, corrected.}
\end{subfigure}
	\caption{Average vehicle lengths along the Tripwire calculated from a the same Frame-strip before and after calibration correction. After the correction, average sizes are distributed uniformly along the Tripwire.(The light blue line indicates the estimated slope of the length values.)\label{fig:calibrations2}}
\end{figure}
%----------------------------------------------------------------------------
\section{Speed testing}
%----------------------------------------------------------------------------
- Speed criteria to meet:
	- minimum speed \SI{25}{frame/s}
	- loosing frames and thus vehicles
	- the optimum is higher, \SI{30}{frame/s}
	- important to be higher for the integration of new functions, processing steps, or extending existing ones
	- e.g. shadow-detection feature, or a more sophisticated classifier

\subsection{Results}
