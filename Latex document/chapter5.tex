%----------------------------------------------------------------------------
\chapter{Experimental results}\label{chap:Tests}
%----------------------------------------------------------------------------
In the following chapter the results of the system tests will be discussed. 
Performance criteria 
%----------------------------------------------------------------------------
\section{Evaluation results}
%----------------------------------------------------------------------------
\subsection{Test videos and }


\subsection{Results}
\subsubsection{Initial setting}
- minSize: 250
- lanes: defined for most videos

\subsubsection{The minimal size of blobs redefined}
- minSize: 200
- lanes: defined for most videos

\subsubsection{No lanes defined}
- minSize: 200
- lanes: no lanes

\subsubsection{Split size added}
- minSize: 100
- splitMinSize : 100

\subsection{Common errors and solutions}
\subsubsection{False positives}
\subsubsection{False negatives}
\subsubsection{Misclassification errors}

%----------------------------------------------------------------------------
\section{Calibration correction}\label{chap:cal_corr}
%----------------------------------------------------------------------------
In the Raw Frames, 
in the image-domain vehicle lengths depend on the distance from the camera caused by perspective deformation
- since vehicle lengths are the basis of classification, perspective distortion must be transformed back/eliminated
- a calibration rectangle based pespective compensation method was used, that is available in the OpenCV library \cite{PersComp}

- the rectangle is defined manually int the Project Configurator during the calibration process presented in section \ref{chap:calibration}
- the calibration rectangle is drawn and its sides' lengths are estimated based on any occurring road lanes, or simply on naked eye examination
- the inaccuracy of human perception
- causes a significant uncertainty
- and empirical analysis shows that errors are actually present in many hand-calibrated videos

- although automatizing this process would help to avoid manual calibration, in heterogeneous environment it is a complex problem and is not implemented in the current version of the system

- to correct the existent settings, the miscalibration can be evaluated examining the length of the vehicles depending on the distance from the camera
- but the manual calibration method can be corrected during an evaluation phase

\subsection{The miscalibration problem}
\subsubsection{Perspective compensation method}
- calibration rectangle that is a square in the ??? space
- distorted on the image, looking like a trapeze or a general rectangle
- the four corners are projected back into a square, and the other pixels are perspectively
- using a transformation map  (OpenCV algorithm)
- as a result, further from the camera the lengths remain the same, and can be correctly classified

\subsubsection{Miscalibrated streams}
- if the edges of the rectangle do not fit a perfect square in reality (valóság)
- the vehicles on the compensated frames are not the same length
- in most cases their lengths still decrease as (távolodva) from the camera
- easily (felismerhető) (jelenség): the int the farest?? road lane cares are classified as bicycles

\subsection{Solutions}
\begin{enumerate}
	\item  \textbf{Estimate decrease}
	\item  \textbf{Calculate proper size ratios}
	\item  \textbf{Redraw calibration rectangle}
\end{enumerate}
- quantify the decrease of the lengths
- based on the correct lengths, redraw the calibration rect

\subsection{Calibration correction results}

%----------------------------------------------------------------------------
\section{Speed testing}
%----------------------------------------------------------------------------
- Speed criteria to meet:
	- minimum speed \SI{25}{frame/s}
	- loosing frames and thus vehicles
	- the optimum is higher, \SI{30}{frame/s}
	- important to be higher for the integration of new functions, processing steps, or extending existing ones
	- e.g. shadow-detection feature, or a more sophisticated classifier

\subsection{Results}
