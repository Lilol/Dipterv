%----------------------------------------------------------------------------
\chapter*{Introduction}\addcontentsline{toc}{chapter}{Introduction}
%----------------------------------------------------------------------------
\section{Traffic load estimation}
%----------------------------------------------------------------------------
In recent years, there has been an increased scope for the automatic analysis of urban traffic activity\ref{Buch2011}.
This case is due to the increasing number of vehicles, the need for better understanding traffic dynamics for investment purposes, and also the accessibility of various sensors.
Formerly traffic estimation was performed by operators, by the means of on-the-spot observation or video analysis.
By the automation of these processes the main concept of traffic analysis is to aid or fully avoid these human operators.

The information obtained by traffic monitoring systems are widely used to regulate traffic flow and contribute to administrative decisions about transportation infrastructure development, maintenance and investments \ref{MagyarKozut}.
In addition to the detection and classification of road users, which is an elementary task, several other monitoring objectives can be supported. 
Typical traffic data such as flow, density, and average speed can be extracted.
Some systems also recognize various situations including traffic violations or accidents (e.g., cars, motorbikes, and pedestrians).
Other real-time applications notify users about congestions, roadworks and accidents, or control traffic lights in order to influence traffic flow direction \ref{AzoSensor, Thiruverahan2015, Ghazal2016}.
It is also possible to estimate environmental measures such as greenhouse gases, pollutant emissions, and fuel consumption in real time using high-resolution video traffic data combined with instantaneous emission models\ref{Morris2012a}. 
In the long term surveillance systems will be integral parts of the smart city infrastructure, identifying and influencing these city's traffic dynamics\ref{enLight}.

\subsection{Traffic monitoring techniques}
Nowadays a variety of sensing modalities is available for on-road vehicle detection.

Inductive loop systems installed in the road are considered the most reliable traffic classification and detection method available, larger metal objects can be accurately detected this way, on the other hand the installation and maintenance is time-consuming and difficult, and slow or temporarily stopped vehicles cannot be detected\ref{Diamond, Zhang2016}.

Other robust detection methods including radar \ref{DeepBlue}, infrared \ref{Swarco, Hussain1995, Ghazal2016} and laser \ref{SICK, Gallego2009} sensors are also popular, although a complex machinery needs to be deployed.

Weight-based classification and speed detection is also possible using a piezoelectric sensor\ref{Te, Rivas2017}.

Video cameras have been installed for a long time for traffic monitoring purposes, because they provide a rich information source for human understanding with a relatively low cost of deployment\ref{Tian2011, Buch2011, VideoSurveillance, LaSemaforica}.
Nowadays camera based sensors became even cheaper, smaller, and of higher quality than ever before, and video-based surveillance is becoming the most popular form of vehicle detection.
Some sensor networks use a digital backend system where the captured data is transferred, here computation and storage takes place.
Since computing power has dramatically increased in the recent years a new generation of visual surveillance systems has emerged: embedded sensors are capable of on-board high-level video processing with limited resources for computation, memory and power \ref{Bramberger2004}.
Some of these platforms have the ability to reliably detect and track on-road vehicles in real time, since they utilize the new image-processing paradigms and advanced hardware: parallelization, multicore processing and graphical processing units (GPUs)\ref{Sivaraman2013}.

Hybrid systems use sensor fusion to combine the data that results from the approaches outlined above\ref{Swarco}.

\subsection{Some results in video-based traffic monitoring}

In addition to the detection and classification of road users, which is an elementary task, some advanced sensing systems can track, measure the velocity, or understand the complex behaviour of the bypassing vehicles.
Automatic license plate recognition is also a extensively used\ref{Sivaraman2013}. 

 Furthermore, in recent years, we have
seen the emergence of computing platforms geared toward

With advances in camera sensing and computational technologies,
advances in vehicle detection using monocular vision,
stereo vision, and sensor fusion with vision have been
an extremely active research area in the intelligent vehicles
community. On-road vehicle tracking has been also extensively
studied. I [3], [5].


In urban environments, several monitoring objectives can
be supported by the application of computer vision and pattern
recognition techniques, including the detection of traffic
violations (e.g., illegal turns and one-way streets) and the
identification of road users (e.g., vehicles, motorbikes, and
pedestrians).

Video analytics may now provide added value to cameras by automatically
extracting relevant information. This way, computer
vision and video analytics become increasingly important for
intelligent transport systems (ITSs).

One level up, detected vehicles
are associated across frames, allowing for vehicle tracking.
%----------------------------------------------------------------------------
\section{The SOLSUN project}
%----------------------------------------------------------------------------
\subsection{Project goals and constraints}
%precision, speed constraints, SW, HW

%----------------------------------------------------------------------------
\section{Thesis goals and organization}
%----------------------------------------------------------------------------
