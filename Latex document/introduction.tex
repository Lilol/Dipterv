%----------------------------------------------------------------------------
\chapter*{Introduction}\addcontentsline{toc}{chapter}{Introduction}
%----------------------------------------------------------------------------
\section{Traffic load estimation}
%----------------------------------------------------------------------------
In recent years, there has been an increased scope for the automatic analysis of urban traffic activity\ref{Buch2011}.
This case is due to the increasing number of vehicles, the need for better understanding traffic dynamics for investment purposes, and also the accessibility of various sensors.
Formerly traffic estimation was performed by operators, by the means of on-the-spot observation or video analysis.
By the automation of these processes the main concept of traffic analysis is to aid or fully avoid these human operators.

The information obtained by traffic monitoring systems are widely used to regulate traffic flow and contribute to administrative decisions about transportation infrastructure development, maintenance and investments \ref{MagyarKozut}.
In addition to the detection and classification of road users, which is an elementary task, several other monitoring objectives can be supported. 
Typical traffic data such as flow, density, and average traffic speed can be extracted.
Some systems also recognize various situations including traffic violations or accidents (e.g., cars, motorbikes, and pedestrians).
Other real-time applications notify users about congestions, roadworks and accidents, or control traffic lights in order to influence traffic flow direction \ref{AzoSensor, Thiruverahan2015}.

!!!! there is now strong interest in environmental factors such as greenhouse gases, pollutant emissions, and fuel consumption. It is now possible to combine high-resolution real-time traffic data with instantaneous emission models to estimate these environmental measures in real time. \ref{Morris2012a}

In the long term surveillance systems will be integral parts of the intelligent city infrastructure, identifying and influencing these city's traffic dynamics\ref{enLight}.

\subsection{Traffic monitoring techniques}
Nowadays a variety of sensing modalities is available for on-road vehicle detection.

Inductive loop systems installed in the road are considered the most reliable traffic classification and detection method available, larger metal objects can be accurately detected this way, on the other hand the installation and maintenance is time-consuming and difficult\ref{Diamond}.

Other robust detection methods including radar \ref{DeepBlue}, infrared \ref{Swarco} and laser \ref{SICK, Gallego2009} sensors are also popular, although a complex machinery needs to be deployed.

Weight-based classification and speed detection is also possible using a piezoelectric sensor\ref{Te, Rivas2017}.

Az ipar számos megoldást kínál erre a problémára. Elterjedtek például az útba telepíthet˝o,
induktív hurkos rendszerek. Ezek pontosan azonosítanak nagyobb fémtárgyakat, ugyanakkor utólagos
beszerelés esetén az úttest felbontását igényik, így karbantartásuk is nehéz [1]. A radaros [2],
infravörös fénnyel [3] vagy lézerrel [4] m˝uköd˝o szenzorok is népszer˝uek, ehhez azonban bonyolult
berendezéseket kell telepíteni. Esetenként a járm˝uvek súlyát használják detektálásra, például
piezo-kristályokon alapuló tömegméréssel – ehhez azonban szintén az úttest megbontására van
szükség [5].

\subsection{Some results in video-based traffic monitoring}
Video cameras have been deployed for a long time for traffic monitoring purposes, because they provide a rich information source for human understanding\ref{Buch2011}.
In addition to the detection and classification of road users, which is an elementary task, some advanced sensing systems can track, measure the velocity, or understand the complex behaviour of the bypassing vehicles.
Automatic license plate recognition is also a extensively used\ref{Sivaraman2013}. 

 Imaging technology has immensely progressed
in recent years. Cameras are cheaper, smaller, and of higher
quality than ever before. Concurrently, computing power has
dramatically increased. Furthermore, in recent years, we have
seen the emergence of computing platforms geared toward
parallelization, such as multicore processing and graphical processing
units (GPUs). Such hardware advances allow computer
vision approaches for vehicle detection to pursue real-time
implementation.
With advances in camera sensing and computational technologies,
advances in vehicle detection using monocular vision,
stereo vision, and sensor fusion with vision have been
an extremely active research area in the intelligent vehicles
community. On-road vehicle tracking has been also extensively
studied. It is now commonplace for research studies to report
the ability to reliably detect and track on-road vehicles in real
time, over extended periods [3], [5].


In urban environments, several monitoring objectives can
be supported by the application of computer vision and pattern
recognition techniques, including the detection of traffic
violations (e.g., illegal turns and one-way streets) and the
identification of road users (e.g., vehicles, motorbikes, and
pedestrians).

Video analytics may now provide added value to cameras by automatically
extracting relevant information. This way, computer
vision and video analytics become increasingly important for
intelligent transport systems (ITSs).

One level up, detected vehicles
are associated across frames, allowing for vehicle tracking.
%----------------------------------------------------------------------------
\section{The SOLSUN project}
%----------------------------------------------------------------------------
\subsection{Project goals and constraints}
%precision, speed constraints, SW, HW

%----------------------------------------------------------------------------
\section{Thesis goals and organization}
%----------------------------------------------------------------------------
