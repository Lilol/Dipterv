%----------------------------------------------------------------------------
\chapter*{Introduction}\addcontentsline{toc}{chapter}{Introduction}
%----------------------------------------------------------------------------
\section{Traffic load estimation}
%----------------------------------------------------------------------------
In recent years, there has been an increased scope for
the automatic analysis of urban traffic activity\ref{Buch2011}.
This case is due to the increasing number of vehicles, the need for better understanding traffic dynamics for investment purposes, and also the
accessibility of various sensors.
Formerly traffic estimation was performed by operators, by the means of on-the-spot observation or video analysis.
By the automation of these processes the main concept of traffic analysis is to aid or fully avoid these human operators.

The information obtained by traffic sensors are widely used to regulate traffic flow, contribute to administrative decisions about transportation infrastructure development, maintenance and investments \ref{MagyarKozut}.
Several other monitoring objectives can be supported. 
Some systems can recognize various situations including traffic
violations and accidents, and classify vehicles (e.g., cars, motorbikes, and
pedestrians).
Real-time applications can be used to notify users about congestions and accidents, or to control traffic lights in order to influence traffic flow direction \ref{AzoSensor}.

In the long term traffic surveillance systems will be integral parts of the infrastructure of intelligent cities, contributing to the identification, observation and interference of these city's traffic dynamics \ref{enLight}.


%In addition, the advancement of analytical techniques
%for processing the video (and other) data, together with increased computing power, has enabled new applications. 
%We define video analytics as computer-vision-based surveillance
%algorithms and systems to extract contextual information from
%video. 


\subsection{Goals and techniques}

\subsection{Some results in video-based traffic monitoring}
A variety of sensing modalities has become available for
on-road vehicle detection, including radar, lidar, and computer
vision. Imaging technology has immensely progressed
in recent years. Cameras are cheaper, smaller, and of higher
quality than ever before. Concurrently, computing power has
dramatically increased. Furthermore, in recent years, we have
seen the emergence of computing platforms geared toward
parallelization, such as multicore processing and graphical processing
units (GPUs). Such hardware advances allow computer
vision approaches for vehicle detection to pursue real-time
implementation.
With advances in camera sensing and computational technologies,
advances in vehicle detection using monocular vision,
stereo vision, and sensor fusion with vision have been
an extremely active research area in the intelligent vehicles
community. On-road vehicle tracking has been also extensively
studied. It is now commonplace for research studies to report
the ability to reliably detect and track on-road vehicles in real
time, over extended periods [3], [5].

Video cameras have been deployed for a long
time for traffic and other monitoring purposes, because they
provide a rich information source for human understanding.
In urban environments, several monitoring objectives can
be supported by the application of computer vision and pattern
recognition techniques, including the detection of traffic
violations (e.g., illegal turns and one-way streets) and the
identification of road users (e.g., vehicles, motorbikes, and
pedestrians).

Video analytics may now provide added value to cameras by automatically
extracting relevant information. This way, computer
vision and video analytics become increasingly important for
intelligent transport systems (ITSs).
%----------------------------------------------------------------------------
\section{The SOLSUN project}
%----------------------------------------------------------------------------
\subsection{Project goals and constraints}
%precision, speed constraints, SW, HW

%----------------------------------------------------------------------------
\section{Thesis goals and organization}
%----------------------------------------------------------------------------
