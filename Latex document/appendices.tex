%----------------------------------------------------------------------------
\appendix
%----------------------------------------------------------------------------
\chapter*{Appendices}\addcontentsline{toc}{chapter}{Appendices}\label{chap:Appendices}
\setcounter{chapter}{1}  % a fofejezet-szamlalo az angol ABC 6. betuje (F) lesz
\setcounter{equation}{0} 
\numberwithin{equation}{section}
\numberwithin{figure}{section}
\numberwithin{lstlisting}{section}
%\numberwithin{tabular}{section}

%----------------------------------------------------------------------------
\section{Advertisement application}
%----------------------------------------------------------------------------
The advertisement application is an alternative utilization of the Traffic Sensor's functionality, availing of that the Sensor is able to detect and count any moving objects in its environment.

The goal of the application is to estimate the number of people present on the corridor of the \viktanszek, where the Sensor will be deployed.
The Traffic Sensor system will be connected to a screen with a web interface, that shows miscellaneous advertisements.

Advertisers will be able to buy time-intervals for their advertisements for different prices.
The intervals' prices are assessed based on the count of people present at that time.

The Traffic Sensor is already fit for the task, however the following settings were modified for the changed situation.
\begin{enumerate}
	\item The camera angle is horizontal, thus the Tripwire is set to vertical.
	
	\item Since the bypassing people are close to the camera, they are not subject to significant  perspective distortion, and there is no need for perspective compensation.
	
	\item The other consequence of the closeness of the camera is that moving objects fill most of the frame, and to detect them, only a rather small Tripwire is necessary, approximately a \SI{20}{pixel} long section.
	An advantage resulting from this is the low computation time.
	
	\item The minimal size of blobs is adjusted to the size of the detected objects empirically.
	
	\item There is no need for classification, since it is highly likely, that on the corridor only people are present.
\end{enumerate}

Although the Advertisement Application is still in pilot phase, and is not fully deployed yet, some example images of the configuration process are seen in figures \ref{fig:adv_app_config}., \ref{fig:adv_app_full_orig}., \ref{fig:adv_app_counting}.

\begin{figure}[!h]
	\centering
	\includegraphics[width=0.5\textwidth]{ad_app_config.png}
	
	\caption[Initial configuration settings of the Advertisement Application]{The GUI of the Project Configurator, showing the initial configuration of the Advertisement Application. Here, the Tripwire crosses the whole frame, although later it is decreased, since the application can count people using a much smaller portion of the line.  \label{fig:adv_app_config}}
\end{figure}

\begin{figure}[!h]
	\centering
	\includegraphics[width=0.6\textwidth]{orig_full.png}
	
	\caption[Test result of the Advertisement Application. (Whole height of the image is used.)]{The result of testing the application. Part of the Original Timeline Image showing a series of people walking in front of the camera. Since the person occupies the whole height of the image, for detection only a smaller section of the image is enough, as seen on figures \ref{fig:adv_app_counting}.}
\end{figure}

\begin{figure}[!h]
	\centering
	\begin{subfigure}[t]{0.6\textwidth}
	\includegraphics[width=\textwidth]{original_section.png}
	\subcaption{Part of the Original Timeline Image.}
	\end{subfigure}
	\hfill
	\begin{subfigure}[t]{0.6\textwidth}
		\includegraphics[width=\textwidth]{mog2_section.png}
		\subcaption{Part of the MOG2 Timeline Image.}
	\end{subfigure}

	\caption[Test result of the Advertisement Application. (Only a small Tripwire section is used)]{Timeline Images of the Advertisement Application using a smaller Tripwire. The MOG2 image captures the moving object correctly each time, but also some noise due to light changes.\label{fig:adv_app_counting}}
\end{figure}

%----------------------------------------------------------------------------
\clearpage\section{VideoMixer application}
%----------------------------------------------------------------------------
The VideoMixer application was created as a side-project for the customers of the Traffic Sensor.

The software is responsible for the visualization of multiple video streams at the same time for demonstration purposes.

The videos can either be recorded from a camera or read from a file.
The user can control, which videos are seen at a certain time, that are blended by the application on top of each other.
The speed of the playing and the looping of the videos can be set as well.

\begin{figure}[!h]
	\centering
	\includegraphics[width=0.5\textwidth]{vm_menu.png}
	
	\caption[Menu of the VideoMixer]{The menu of the VideoMixer. Whenever a key is pressed, it is displayed, as well as possible control options. \label{fig:video_mixer_menu}}
\end{figure}

The settings (e.g the name and type of the input videos, the controlling keys and the whether the videos are looped) are defined in a configuration file, similarly to the Traffic Sensor's (seen in code-listing \ref{lst:config_vm}).

The application can be utilized as a demonstration of the Traffic Sensor's operation as well, some example frames are figures \ref{fig:video_mixer_frames}., \ref{fig:video_mixer_menu}..

\begin{lstlisting}[frame=single,float=!ht,caption={ Part of a configuration file for the VideoMixer application. The file sets parameters for the main input video {(section [mainVideoInput])}, and the overlay videos {(section [overlayVideos])}, that are blended on top. The file defines whether the video is read from a camera or a file, whether specific parts of the background are removed or not, and other adjustments, like the comment that is displayed on the screen, whenever a video starts playing.},label=lst:config_vm]
{
[mainVideoInput]
cameraInput = false
cameraIndex = 0
filename = orig.avi
loop = true
removeBackgroundColor = 0x00ff00

[overlayVideos]
filename_0 = mog2.avi
gpioNum_0 = 0
overlayRatio_0 = 0.4
comment_0 = playing file 1
displayCommentOnScreen_0 = false
removeBackgroundColor_0 =  0x00ff00
loop_0 = true
filename_1 = speed.avi
...
\end{lstlisting}

\begin{figure}[!h]
	\centering
	\begin{subfigure}[t]{0.6\textwidth}
		\includegraphics[width=\textwidth]{orig_mog22.png}
		\subcaption{The overlay video captures MOG2 Frames and the building process of a MOG2 Timeline Image.}	
		\vspace{0.3cm}
	\end{subfigure}
	\begin{subfigure}[t]{0.6\textwidth}
		\includegraphics[width=\textwidth]{orig_result.png}
		\subcaption{The overlay video captures a Result Timeline image, that shows the detected vehicles with different colors projected on top of the main video. The Result Timeline also displays vehicle parameters, such as size and type. This way both correct and faulty detections can be analysed.}
		\vspace{0.3cm}
	\end{subfigure}
	
	\begin{subfigure}[t]{0.6\textwidth}
		\includegraphics[width=\textwidth]{orig_speeds.png}
		\subcaption{The blended video is a Speed Timeline. The direction of each detected vehicle can be read by the colours of the Speed Timeline (the vehicles moving from the right to the left are marked with red, and the ones heading in the other direction are colored with blue). }
	\end{subfigure}

	\caption[Some frames from the VideoMixer application]{Some frames created with the VideoMixer application. The overlay video is blended on top of the main video, that consists of an Original Frame and an Original Timeline, and was recorded using the Traffic Sensor. Overlay videos are other Timeline Images, including MOG2, Speed and Result.\label{fig:video_mixer_frames}}
\end{figure}

%----------------------------------------------------------------------------
\clearpage\section{Some examples of the output Timeline Images}\label{app:TIs}
%----------------------------------------------------------------------------
This appendix presents some test results in the form of output Timeline Images, focusing on remarkable points and interesting details.

\vspace{1cm}
\centering
\begin{figure}[h!]
	\centering
	\captionsetup{width=0.9\textwidth,aboveskip=10pt}
	\includegraphics[width=\textwidth]{GoldA1.png}
	\caption[Some output Timelines of the GoldA1 test video]{The figures (from the top to the bottom) are as follows: Original Timeline Image, MOG2 Timeline Image, Size Timeline Image, Speed Timeline Image and Evaluation Result Timeline Image respectively. \\
	On night-time videos like GoldA1 a prevalent error cause is that headlights of vehicles, that are reflected back by the road lead to false alarms (some examples are on the right side of the image). Headlight blobs fusing with the following vehicles are also sources of misclassification errors, since the vehicle size increases (seen on the Size Timeline). \\
	One can notice that the colours on the Speed Timeline  mark the directions of the road lanes. \label{fig:GoldA1}}
\end{figure}

\centering
\begin{figure}[h!]
	\centering
	\captionsetup{width=0.9\textwidth,aboveskip=10pt}
	\includegraphics[width=0.8\textwidth]{GoldB12.png}
	\caption[Some output Timelines of the GoldB1 test video]{The figures (from the top to the bottom) are as follows: Original Timeline Image, MOG2 Timeline Image, Motionless Timeline Image, Size Timeline Image, Speed Timeline Image and Evaluation Result Timeline Image respectively.\\
	 The following common error cases are seen in the Evaluation Result Timeline Image. Small blobs of motorcycles are filtered out (in the middle crossed with red) generating false negative detections. Some stopping vehicles add noise to the detection when missed by the fusing algorithm (false negatives caused by noise are on the left side of the image). The splitting mechanism tend to erroneously cut vehicles, that have a peculiar shape (in the middle and in the right). \\
	 One can notice, that the Parameter Timelines were calculated without lanes defined, since the whole outlines of blobs are filled with values. Intensities on the Size Timeline shows well the approximate length of vehicles. \label{fig:GoldB1}}
\end{figure}

\clearpage
\centering
\begin{sidewaysfigure}[h!]
	\centering
	\captionsetup{width=0.9\textwidth,aboveskip=10pt}
	\includegraphics[width=\textwidth]{ErE5.png}
	\caption[Some output Timelines of the ErE5 test video]{The figures (from the top to the bottom) are as follows: Original Timeline Image, MOG2 Timeline Image, Size Timeline Image, Following Distance Timeline Image, Speed Timeline Image and Result Timeline Image respectively.\\
    The video was recorded at the IBikeBudapest cyclist demonstration, therefore road users are almost exclusively cyclists. \\
    Comparing the Size and the Following Distance Timeline, it is observable, that following distance and length are calculated in alternate times, whether the Tripwire is empty (following distances) or occupied by a vehicle (length). An interesting point of the result is cars appearing in red on the top part of the Speed Timeline, heading in the opposite direction as the cyclists. Different coloured dots inside the vehicles in the Speed Timeline are caused by noise, that is the result of the size changes of the vehicle (further details on parameter calculation based on vehicle position is in chapter \ref{chap:parameter_evaluation}.). \label{fig:ErE5}}
\end{sidewaysfigure}

\centering
\begin{sidewaysfigure}[h!]
	\centering
	\captionsetup{width=0.9\textwidth,aboveskip=10pt}
	\includegraphics[width=\textwidth]{Fovam_C.png}
	\caption[Some output Timelines of the Fovam\_C test video]{The figures (from the top to the bottom) are as follows: Original Timeline Image, MOG2 Timeline Image and Result Timeline Image respectively. \\
	The video was recorded at sundown, and captures only a few cars. Although the environmental light frequently changes (an example is the lilac section on the left on the Original Timeline), the count of false detections is low. Fused cars in the middle are split appropriately. However the stopping pedestrian (top right corner) is ignored, al well as the other two pedestrians who are falsely split based on their shapes.}
\vspace{0.5cm}
	\centering
	\captionsetup{width=0.9\textwidth,aboveskip=10pt}
	\includegraphics[width=\textwidth]{ErC1.png}
	\caption[Some output Timelines of the ErC1 test video]{The figures (from the top to the bottom) are as follows: Original Timeline Image, Result Timeline Image and Evaluation Result Timeline Image respectively. \\
	The video captures a stopping tram and a handful of cars and pedestrians. The tram's slow movement still challenges the fusing algorithm, thus causes some noise.}%{ErC kódú videó futtatási eredménye. Fentről lefelé a képek rendre a következők. entről lefelé a képek rendre a következők. Eredeti videóból képzett TK. MOG2 TK. A kiértékelés eredménye. \\
		%A videón egy megálló villamos, illetve a mellette haladó gyalogosok és autók láthatók. A rendszer jól összeilleszti a megálló villamos blobját, de helytelenül a jobb alsó sarokban található blobokat (hosszú téglalap), amely csak a a háttér folytonossága miatt tűnik a rendszernek megálló autónak.}
\end{sidewaysfigure}

\centering
\begin{sidewaysfigure}[h!]
	\centering
	\captionsetup{width=0.9\textwidth,aboveskip=10pt}
	\includegraphics[width=\textwidth]{SasA1.png}
	\caption[Some output Timelines of the SasA1 test video]{The figures (from the top to the bottom) are as follows: Original Timeline Image, MOG2 Timeline Image, Speed Timeline Image and Evaluation Result Timeline Image respectively. \\
	The software performs well on this video. Most fused vehicles are correctly split, and stopping vehicles' blobs are reattached as well. Due to shaky camera, that causes the background to change, some false alarms are present (on the top, middle part). The Speed Timeline is calculated only in the centre of road lanes displaying fewer values than the Parameter Timelines in figures \ref{fig:ErE5}., \ref{fig:GoldA1}., \ref{fig:GoldB1}.}
\end{sidewaysfigure}

\clearpage
\begin{sidewaysfigure}[h!]
	\centering
	\captionsetup{width=0.9\textwidth,aboveskip=10pt}
	\includegraphics[width=\textwidth]{MisclassificationErrorTimeline_GoldA2.png}
	\caption[Misclassification Error Timeline]{GoldA2 METI}
	\vspace{0.5cm}
	\centering
	\captionsetup{width=0.9\textwidth,aboveskip=10pt}
	\includegraphics[width=\textwidth]{FalsePositiveErrorTimeline_Fovam_C.png}
	\caption[False Positive Error Timeline]{FovamC FPETI}
	\vspace{0.5cm}
	\centering
	\captionsetup{width=0.9\textwidth,aboveskip=10pt}
	\includegraphics[width=\textwidth]{FalseNegativeErrorTimeline_SasadA1_Sun_heavy.png}
	\caption[False Negative Error Timeline]{SasadA1 FNETI}
\end{sidewaysfigure}
