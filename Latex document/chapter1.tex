%----------------------------------------------------------------------------
\chapter{Basic concepts and terms}\label{chap:Concepts}
%----------------------------------------------------------------------------
In the following chapter certain basic definitions and ideas necessary for understanding the operation of the Traffic Sensor will be discussed and formalized.

Let the intensities of each pixel of a given frame be denoted by $I(x,y,t)$, where $x, y$ are the variables of the discrete spatial axes, and $t$ notes the discrete time:
\begin{displaymath}
x=0,1,2,\dotsc,n_{cols-1}
\end{displaymath} 
,
\begin{displaymath}
y=0,1,2,\dotsc,n_{rows-1}
\end{displaymath}
,
\begin{displaymath}
t=0,1,2,\dotsc,n_{frames-1}
\end{displaymath}.

$I$ is a three-element vector consisting of the intensities of each channel of an RGB image.

\section{Tripwire}
%----------------------------------------------------------------------------
The tripwire is a straight line in the frame, where the detection takes place.
Let the intensities of the pixels of the tripwire be denoted by $T(x,y)=T(k)$, where $x$ and $y$ form a 4-connected line on the image pane, and $k$ means the series of the successive pixels of the line:
\begin{displaymath}
	k=0,1,\dotsc,l_{line}
\end{displaymath},
where $l_{line}$ means the length of the tripwire, that is the number of points in the line.
An example of a tripwire is depicted in figure \ref{fig:tripwire}.
The intensities of tripwire-pixels on the $t$th frame is: $T(x,y,t)$.
For detection purposes the tripwire must be approximately perpendicular to the traffic direction.
To ensure this angle the tripwire is configured manually in the Traffic Sensor application .
In recent scholarly articles the tripwire is mostly called either virtual detection line (VDL) or check-line.
%----------------------------------------------------------------------------
\section{Timeline image}
%----------------------------------------------------------------------------
A timeline image (TI) is a sequence of the the states of the tripwire on successive frames.
The horizontal axis of the TI represents time, and the vertical axis corresponds to the pixels of the tripwire itself. 
In other words, the $i$th column in the timeline image is filled with the tripwire pixels of the $i$th time frame.
The creation process of a TI is depicted in figure \ref{fig:TI_creation}.
%----------------------------------------------------------------------------
\section{Frame-strip}
%----------------------------------------------------------------------------
A frame-strip is a sequence of frames in the order of recording.

%--------------------------------------------------------------------------
