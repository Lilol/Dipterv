%----------------------------------------------------------------------------
\chapter{Basic concepts and terms}\label{chap:Concepts}
%----------------------------------------------------------------------------
In the following chapter I discuss and formalize certain basic definitions and ideas necessary for understanding the operation of the Traffic Sensor.

Let the intensities of each pixel of a given frame be denoted by $I(x,y,t)$, where $x, y$ are the variables of the discrete spatial axes, and $t$ notes the discrete time, where:
\begin{displaymath}
x=0,1,2,\dotsc,n_{cols-1}
\end{displaymath} 
,
\begin{displaymath}
y=0,1,2,\dotsc,n_{rows-1}
\end{displaymath}
,
\begin{displaymath}
t=0,1,2,\dotsc,n_{frames-1}
\end{displaymath}.

$I$ is a three-element vector consisting of the intensities of each channel of an RGB image.

\section{Tripwire}
%----------------------------------------------------------------------------
The tripwire is a straight line in the frame, where the detection takes place.
Let the intensities of the pixels of the tripwire be denoted by $T(x,y)$, where $x$ and $y$ form a 4-connected line on the image pane.
An example of a tripwire is depicted in figure \ref{fig:tripwire}.
The intensities of tripwire on the $t$. frame is: $T(x,y,t)$.
For detection purposes the tripwire must be approximately perpendicular to the traffic direction.
In the discussed application to ensure this angle the tripwire is configured manually.
In recent scholarly articles the tripwire is mostly called either virtual detection line (VDL) or check-line.
%----------------------------------------------------------------------------
\section{Timeline image}
%----------------------------------------------------------------------------
the horizontal axis represents the spatial
distribution on the check-line, and the vertical axis indicates
the temporal distribution. In other words, the ith row in the
profile image is filled with the extracted foreground on the
check-line in the ith time frame
%----------------------------------------------------------------------------
\section{Framestrip}
%----------------------------------------------------------------------------

%--------------------------------------------------------------------------
