%----------------------------------------------------------------------------
\chapter{Basic concepts and terms}\label{chap:Concepts}
%----------------------------------------------------------------------------
In the following chapter certain basic definitions and ideas necessary for understanding the operation of the Traffic Sensor will be discussed and formalized.

Let one pixel of a red-green-blue (RGB) image be denoted by $\boldsymbol{I}(x,y,t)$, where $x, y$ are the variables of the discrete spatial axes, and $t$ notes the discrete time.
$\boldsymbol{I}(x,y,t)$ is a three-element vector consisting of the intensities of each channel of an RGB image.

\begin{figure}[h!]
	\centering
	\scalebox{.7}{\input{figures/notations.pdf_tex}}
	\label{fig:notations}
	\caption{No??.}
\end{figure}

\section{Frame}
Let a frame at a given $t$ time be denoted by
\begin{displaymath}
	\boldsymbol{F}_t=\boldsymbol{F}_t(x,y),\\
	x=0,1,2,\dotsc,n_{\text{cols}}-1,\\
	y=0,1,2,\dotsc,n_{\text{rows}}-1,
\end{displaymath}
where $n_{\text{cols}}$ stands for the number of columns, and $n_{\text{rows}}$ is the number of rows in a frame. 

\section{Frame-strip}
%----------------------------------------------------------------------------
A frame-strip is a sequence of frames in the order of recording, denoted by:
\begin{displaymath}
	\boldsymbol{FS}=\boldsymbol{FS}(t)=\boldsymbol{F}_t(x,y),
	t=0,1,2,\dotsc,n_{\text{frames}}-1,
\end{displaymath}
where $n_{\text{frames}}$ denotes the frame count in a video sequence.
%--------------------------------------------------------------------------
\section{Tripwire}
%----------------------------------------------------------------------------
The tripwire is a straight line in the frame, where the detection takes place.
Let the intensities of the tripwire at a given $t$ time be denoted by 
\begin{displaymath}
\boldsymbol{T}_t=\boldsymbol{F}_t(x,y)=\boldsymbol{T}_t(k),
\end{displaymath}
where $x$ and $y$ form a 4-connected line on the image pane of the $t$th frame, and the series of the successive pixels of the line are numbered with $k$:
\begin{displaymath}
	k=0,1,\dotsc,l_{\text{tripwire}}-1,
\end{displaymath}
where $l_{\text{tripwire}}$ stands for the number of pixels of the line, that is the length of the tripwire.

For detection purposes the tripwire must be approximately perpendicular to the traffic direction.
To ensure this angle the tripwire is configured manually in the Traffic Sensor application .
In recent scholarly articles the tripwire is mostly called either virtual detection line (VDL) or check-line.
%----------------------------------------------------------------------------
\section{Timeline image}
%----------------------------------------------------------------------------
A timeline image (TI) is a sequence of the the states of the tripwire on successive frames, denoted by
\begin{displaymath}
	\boldsymbol{TI}=\boldsymbol{TI}(t,k),
\end{displaymath}
where the horizontal axis of the TI represents time ($t$), and the vertical axis corresponds to the pixels of the tripwire itself ($k$).
 
In other words, the $\text{i}$th column in the timeline image is filled with the tripwire pixels of the $\text{i}$th time frame:
\begin{displaymath}
	\boldsymbol{TI}(t=\text{i},k)=\boldsymbol{T}_\text{i},
\end{displaymath}
and the $\text{i}$th row corresponds to the $\text{i}$th tripwire pixel changes throughout the video stream:
\begin{displaymath}
\boldsymbol{TI}(t,k=\text{i})=\boldsymbol{T}(\text{i}),
\end{displaymath}

The creation of a TI is depicted in figure \ref{fig:TI_creation}.
In the literature timeline images are called stack of lines or time-spatial images.

\begin{figure}[bh]
	\centering
	\scalebox{.7}{\input{figures/TI_creation.pdf_tex}}
	\label{fig:TI_creation}
	\caption{The generation procedure of a timeline image. Each column corresponds to a tripwire state. The horizontal axis represents time.}
\end{figure}
%----------------------------------------------------------------------------
